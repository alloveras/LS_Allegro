\documentclass[11pt]{article}

%Afegim els "packages" necessaris per generar el document
\usepackage[utf8]{inputenc}
\usepackage{graphicx}
\usepackage{fancyhdr}
\usepackage{hyperref}
\usepackage{tocloft}
\usepackage[T1]{fontenc}


\title{Instal·lació Allegro 5 (Linux)}
\author{Autor: Albert Lloveras Carbonell\\Revisió: Joaquim Porte Rodríguez}
\date{}

\renewcommand\contentsname{\huge Índex \vspace{8pt} \hrule}
\renewcommand{\cftsecleader}{\cftdotfill{\cftdotsep}}

%------------------------------------------------------------
%	Definició de Capçaleres i Peus de Pàgina
%------------------------------------------------------------

\fancypagestyle{pageStyle}{
	%Definim la capçalera de l'esquerra (Logo Salle)
	\fancyhead[L]{
		\includegraphics[scale=0.25]{img/la_salle_logo.jpg}
	} 

	%Definim la capçalera de la dreta (Info Curs i Assignatura)
	\fancyhead[R]{
		\vtop{
			Manual d'instal·lació\\
			d'Allegro 5 (Ubuntu)\\
		}
	}
	\fancyfoot[C]{} %Peu de pàgina central
	\fancyfoot[L]{Departament d'Enginyeria - Informàtica} %Peu de pàgina esquerra
	\fancyfoot[R]{\thepage} %Peu de pàgina dreta (Número de pàgina)	
	%Configuracions extra
	\renewcommand{\headrulewidth}{0pt} %Ocultem la línia de separació entre 		capçalera i bloc de text
	\renewcommand{\footrulewidth}{2pt} %Fixem la línia de separació entre peu de pàgina i bloc de text
	\setlength{\headheight}{67pt} %Fixem la mida de la capçalera	
	
}

%Definim la macro (nofooter) per poder amagar el peu de pàgina quan interessi
\fancypagestyle{nofooter}{
	\fancyfoot{}
	\renewcommand{\footrulewidth}{0pt}
}

%Definim la macro (nofooter) per poder amagar el peu de pàgina quan interessi
\fancypagestyle{empty}{
	\fancyfoot{}
	\fancyhead{}
	\renewcommand{\footrulewidth}{0pt}
	\renewcommand{\headrulewidth}{0pt}
}

\begin{document}

\pagestyle{empty}
\begin{center}

	%--- Logo de la portada --- %
	\includegraphics[width=0.15\textwidth]{img/allegro.png}~\\[1cm]
	
	%---- Nom del programa --- %
	\textsc{\LARGE Allegro 5}\\[1.5cm]
	
	%-- Nom del manual -- %
	\hrule
	\vspace{8pt}
	\huge{\bfseries Manual d'instal·lació (Ubuntu)}
	\vspace{8pt}
	\hrule	
	\vspace{12pt	}

	% -- Nom de l'autor i del supervisor
	\noindent
	\begin{minipage}{0.4\textwidth}
		\begin{flushleft} \large
			\emph{Autor:}\\
			Albert \textsc{Lloveras}
		\end{flushleft}
	\end{minipage}%
	\begin{minipage}{0.4\textwidth}
		\begin{flushright} \large
			\emph{Revisió:} \\
			Joaquim \textsc{Porte}
		\end{flushright}
	\end{minipage}
	\vfill

\end{center}

%---- Índex ---- %
\newpage

\pagestyle{empty}
\tableofcontents


%---- Començament del document --- %
\newpage
\pagestyle{pageStyle}

\section{Section one}
\section{Section two}
\end{document}